\documentclass[12pt]{article}
\usepackage{amsmath}
\usepackage{amsmath}
\usepackage{amssymb}
\usepackage{amsthm}
\usepackage{amscd}
\usepackage{amsfonts}
\usepackage{dsfont}
\usepackage{fancyhdr}
\usepackage{enumerate}
\usepackage[utf8]{inputenc}
\usepackage[margin=1in]{geometry}
\usepackage{color}
\usepackage{hyperref}


\newcommand{\ket}[1]{|#1\rangle}
\newcommand{\bra}[1]{\langle#1|}
\newcommand{\ketbra}[2]{| #1 \rangle \langle #2 |}
\newcommand{\braket}[2]{\langle  #1 |#2 \rangle}


%THEOREM STYLES
\theoremstyle{plain}
\newtheorem{theorem}{Theorem}
\theoremstyle{definition}
\newtheorem{exerc}{Exercise}
\newtheorem{ans}{Answer}

%NICE HEADER
\pagestyle{fancy}\lhead{\textcolor{red}{Yuri Florentino Kosfeld}} \rhead{CQ - 2025/1}
\chead{{\large{\bf }}} \lfoot{} \rfoot{\bf \thepage} 
\cfoot{}

\newcounter{list}

\begin{document}

\begin{center}
\Large \textbf{{Assignment 1 - due March 30th}}
\end{center} 


\begin{ans}
 bla bla bla
 Examples:
 \begin{enumerate}[(a)]
 	\item We will prove that $U^{*}U = I$. We know $U$ is diagonalizable, so $U$ can be written as \[U = \sum_{i=0}^{n} \lambda_i \ketbra{v_i}{v_i}\] where
			all $\lambda_i$ are eigenvalues and $\ket{v_i}$ are there respective eigenvectors. 
			Then, \[ U^{*} = \sum_{i=0}^{n} \overline{\lambda_i} \braket{v_i}{v_i}\]
			Note that all eigenvectors are orthogonal, each means that, $\braket{v_i}{v_i} \ketbra{v_j}{v_j} = 0$ when $i \neq j$, and $\braket{v_i}{v_i} \ketbra{v_j}{v_j} = 1$
			when $i = j$. So we have, \[ U^{*}U = \sum_{i=0}^{n} \overline{\lambda_i} \lambda_i\] and by the hypothesis
			$\overline{\lambda_i} \lambda_i = \left\lvert \lambda_i \right\rvert = 1$. Therefore $ U^{*}U = I$.
 	\item Take $\lambda$ a eigenvalue of U unitary, and $\ket{v}$ his associated eigenvector, so the equity bellow
			is true, \[ U \ket{v} = \lambda\ket{v} \]
			Note that, $ = \overline{\lambda}\bra{v}$.
			Then, when we multiply the first equation for $\bra{v}U^{*}$ on the left, we get 
			\[ \bra{v}U^{*} U \ket{v} = \bra{v}U^{*} \lambda \ket{v}\]
			\[ \braket{v}{v} = \bra{v}\overline{\lambda}\lambda \ket{v}\]
			\[ \braket{v}{v} = \overline{\lambda}\lambda \braket{v}{v}\]
			By the property of the inner product, we have that $\braket{v}{v} \neq 0$, so we can divide the last
			equation on both sides for $\braket{v}{v}$. And then, 
			\[ 1 = \overline{\lambda}\lambda = \left\lvert \lambda \right\rvert\]
 	\item $\bra u M \ket v$
 \end{enumerate}
 
\end{ans}

\noindent \hrulefill

\begin{ans}
 bla bla bla
\end{ans}

\noindent \hrulefill

\begin{ans}
 bla bla bla
\end{ans}

\noindent \hrulefill

\begin{ans}
 bla bla bla
\end{ans}

\noindent \hrulefill

\begin{ans}
 bla bla bla
\end{ans}

\noindent \hrulefill

\begin{ans}
 bla bla bla
\end{ans}

\noindent \hrulefill

\begin{ans}
 bla bla bla
\end{ans}




\end{document}

